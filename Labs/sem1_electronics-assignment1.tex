%%%%%%%%%%%%%%%%%%%%%%%%%%%%%%%%%%%%%%%%%
% University/School Laboratory Report
% LaTeX Template
% Version 3.1 (25/3/14)
%
% This template has been downloaded from:
% http://www.LaTeXTemplates.com
%
% Original author:
% Linux and Unix Users Group at Virginia Tech Wiki 
% (https://vtluug.org/wiki/Example_LaTeX_chem_lab_report)
%
% License:
% CC BY-NC-SA 3.0 (http://creativecommons.org/licenses/by-nc-sa/3.0/)
%
%%%%%%%%%%%%%%%%%%%%%%%%%%%%%%%%%%%%%%%%%

%----------------------------------------------------------------------------------------
%	PACKAGES AND DOCUMENT CONFIGURATIONS
%----------------------------------------------------------------------------------------

\documentclass{article}

\usepackage[version=3]{mhchem} % Package for chemical equation typesetting
\usepackage{siunitx} % Provides the \SI{}{} and \si{} command for typesetting SI units
\usepackage{graphicx} % Required for the inclusion of images
\usepackage{natbib} % Required to change bibliography style to APA
\usepackage{amsmath} % Required for some math elements 

\usepackage{hyperref}
\hypersetup{
	colorlinks=true,
	linkcolor=blue,
	filecolor=magenta,      
	urlcolor=cyan,
}


%\usepackage{times} % Uncomment to use the Times New Roman font

%----------------------------------------------------------------------------------------
%	DOCUMENT INFORMATION
%----------------------------------------------------------------------------------------

\title{Assignment 1} % Title

\author{Pranav \textsc{Gade}} % Author name

\date{\today} % Date for the report

\begin{document}

\maketitle % Insert the title, author and date

\begin{center}
\begin{tabular}{l r}
Batch: & CS\&AI \\
Roll no.: & LCI2020010 \\
Date Performed: & December 1, 2020 \\ % Date the experiment was performed
Instructor: & Dr. Somesh Kumar % Instructor/supervisor
\end{tabular}
\end{center}

% If you wish to include an abstract, uncomment the lines below
% \begin{abstract}
% Abstract text
% \end{abstract}

%----------------------------------------------------------------------------------------
%	SECTION 1
%----------------------------------------------------------------------------------------
\renewcommand\thesubsubsection{\arabic{subsubsection}}

\section*{Part A}

\subsubsection{What is the difference between a silicon (Si) and a germanium (Ge) semiconductor? Why we prefer Silicon over Germanium?}
\begin{tabular}{|c|c|c|}
	\hline
	& Ge & Si \\
	\hline
	Leakage current & $ \micro A $ & $ \nano A $ \\
	\hline
	$ E_G0 $ & 0.75eV & 1.21eV \\
	\hline
	Operating Temperature & -60C to 75C & -60C to 275C \\
	\hline
	$ \mu_n $ & $ 3800cm^2/V $ sec & $ 1300cm^2/V  sec $ \\
	\hline
	$ \mu_p $ & $ 1800cm^2/V sec $ & $ 500cm^2/V sec $ \\
	\hline
\end{tabular}
We prefer Si over Ge because Si has a lower leakage current, and is suitable over a larger range of operating temperature. Si is also one of the most abundant elements on earth.

\subsubsection{Explain about the Fermi level in p type semiconductor. How the Fermi level in p type semiconductor varies with temperature and doping.}
Fermi level is the maximum energy possessed by electrons at 0K.

%----------------------------------------------------------------------------------------


\end{document}