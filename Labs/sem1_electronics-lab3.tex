%%%%%%%%%%%%%%%%%%%%%%%%%%%%%%%%%%%%%%%%%
% University/School Laboratory Report
% LaTeX Template
% Version 3.1 (25/3/14)
%
% This template has been downloaded from:
% http://www.LaTeXTemplates.com
%
% Original author:
% Linux and Unix Users Group at Virginia Tech Wiki 
% (https://vtluug.org/wiki/Example_LaTeX_chem_lab_report)
%
% License:
% CC BY-NC-SA 3.0 (http://creativecommons.org/licenses/by-nc-sa/3.0/)
%
%%%%%%%%%%%%%%%%%%%%%%%%%%%%%%%%%%%%%%%%%

%----------------------------------------------------------------------------------------
%	PACKAGES AND DOCUMENT CONFIGURATIONS
%----------------------------------------------------------------------------------------

\documentclass{article}

\usepackage[version=3]{mhchem} % Package for chemical equation typesetting
\usepackage{siunitx} % Provides the \SI{}{} and \si{} command for typesetting SI units
\usepackage{graphicx} % Required for the inclusion of images
\usepackage{natbib} % Required to change bibliography style to APA
\usepackage{amsmath} % Required for some math elements 

\usepackage{hyperref}
\hypersetup{
	colorlinks=true,
	linkcolor=blue,
	filecolor=magenta,      
	urlcolor=cyan,
}


%\usepackage{times} % Uncomment to use the Times New Roman font

%----------------------------------------------------------------------------------------
%	DOCUMENT INFORMATION
%----------------------------------------------------------------------------------------

\title{Experiment 3\\P-N Junction Diode} % Title

\author{Pranav \textsc{Gade}} % Author name

\date{\today} % Date for the report

\begin{document}
	
	\maketitle % Insert the title, author and date
	
	\begin{center}
		\begin{tabular}{l r}
			Batch: & CS\&AI \\
			Roll no.: & LCI2020010 \\
			Date Performed: & December 24, 2020 \\ % Date the experiment was performed
			Instructor: & Dr. Somesh Kumar % Instructor/supervisor
		\end{tabular}
	\end{center}
	
	% If you wish to include an abstract, uncomment the lines below
	% \begin{abstract}
	% Abstract text
	% \end{abstract}
	
	%----------------------------------------------------------------------------------------
	%	SECTION 1
	%----------------------------------------------------------------------------------------
	
	\section{Aim}
	
	To study the Volt-Ampere Characteristics of a P-N junction diode.
	
	\section{Objective}
	\begin{enumerate}
		\item To plot the Volt-Ampere characteristics of a P-N junction diode.
		\item To find cut-in Voltage for silicon P-N Junction Diode.
		\item To find static and dynamic resistances of a P-N Junction Diode.
	\end{enumerate}
	
	\subsection{Components Required}
	\textbf{LTSpice Simulator:} P-N Junction Diode(1N914), Resistance(1$\Omega$), Power Supply, and wires are all provided in the software.
	
	%----------------------------------------------------------------------------------------
	%	SECTION 2
	%----------------------------------------------------------------------------------------
	
	\section{Explanation}
	A PN Junction diode is made from p-type and n-type semiconductors. P-type semiconductors are created by adding electron acceptor impurities to a semiconductor during manufacture. Therefore, holes are the majority charge carrier. In n-type semiconductors, electron donor impurities are added. Therefore, the majority charge carriers are electrons. A diode is a device that allows current flow in only one direction and provides very large resistance in opposite direction. \\
	The diode is said to be forward biased when positive terminal of the input is connected to the anode of the diode, and the negative terminal of the input is connected to the cathode of the diode. In this case the voltage drop across the depletion region due to majority carriers is by $ V_0-V_f $. However, minority charge carriers do play a role at low voltages, and therefore the current through the diode isn't 0. So, beyond a certain voltage the diode can be assumed to be a closed switch. This voltage is also referred as the cut-in voltage. \\
	In this condition, the volt-ampere characteristics are given by the equation $ I_f = I_0[ e^{V_d/\eta k T} - 1] $ (where $ I_0 $ is the reverse saturation current; $ V_d $ is the voltage across the diode, $\eta$ is the recombination factor or utility factor (1 for Ge, 2 for Si)). \\
	The diode is said to be reversed biased when the positive terminal of the input is connected to the cathode of the diode and the negative terminal of the input is connected to the anode of the diode. In this case the voltage drop across the depletion region is given by $ V_0 + V_R $. In this case a very small reverse saturation current does flow. This is due to the presence of minority charge carriers. Because the concentration of minority charge carriers is very low, we can assume that this leakage current is constant, unless the voltage applied is very high. Therefore, the diode can be approximated to be a open switch in this case. \\
	
	\section{Circuit Diagrams}
	\subsection{Forward Bias}
		\begin{center}
			\label{schematic:fwd-bias}
			\includegraphics[width=0.7\linewidth]{"forward bias schematic"}
		\end{center}		
	\subsection{Reverse Bias}
		\begin{center}
			\label{schematic:rev-bias}
			\includegraphics[width=0.7\linewidth]{"reverse bias 	schematic"}
		\end{center}		
	\section{Procedure}
	\subsection{Forward Bias}
	\begin{enumerate}
		\item Create a schematic as shown in fig. \ref{schematic:fwd-bias}
		\item Set the simulation software to run a DC sweep from 0V-3V
		\item Run the simulation, and plot the current through the diode $ I_F $ against the voltage across the diode $ V_F $
		\item Find the cut-in voltage and static resistance from the graph.
	\end{enumerate}
	\subsection{Reverse Bias}
	\begin{enumerate}
		\item Create a schematic as shown in fig. \ref{schematic:rev-bias}
		\item Set the simulation software to run a DC sweep from 0V-7V
		\item Run the simulation, and plot the voltage across the diode $ V_R $ and current through the diode $ I_R $ against the source voltage $ V_S $
	\end{enumerate}

	\section{Observation Tables}
	\subsection{Forward Bias}
	\begin{center}
	\begin{tabular}{|c|c|c|c|}
		\hline
		& $ V_s $ (Volts) & $ V_f $ (Volts) & $ I_f $ (mA) \\
		\hline
		1 & 0.25 & 0.25 & 0 \\
		\hline
		2 & 0.5 & 0.5 & 0 \\
		\hline
		3 & 0.75 & 0.73 & 19.7 \\
		\hline
		4 & 1 & 0.876 & 125.5 \\
		\hline
		5 & 1.25 & 0.989 & 263.5 \\
		\hline
		6 & 1.5 & 1.092 & 410.2 \\
		\hline
		7 & 1.75 & 1.19 & 560.6 \\
		\hline
		8 & 2 & 1.287 & 713.1 \\
		\hline
		9 & 2.5 & 1.383 & 1021.5 \\
		\hline
		10 & 2.25 & 1.478 & 866.9 \\
		\hline
		11 & 2.75 & 1.573 & 1176.9 \\
		\hline
		12 & 3.00 & 1.667 & 1332.7 \\
		\hline
	\end{tabular}
	\end{center}
	\subsection{Reverse Bias}
	\begin{center}
	\begin{tabular}{|c|c|c|c|}
		\hline
		& $ V_s $ (Volts) & $ V_R $ (Volts) & $ I_R $ (nA) \\
		\hline
		1 & 0 & 0 & 0 \\
		\hline
		2 & 0.5 & 0.5 & 2.52 \\
		\hline
		3 & 1 & 1 & 2.521 \\
		\hline
		4 & 1.5 & 1.5 & 2.522 \\
		\hline
		5 & 2 & 2 & 2.522 \\
		\hline
		6 & 2.5 & 2.5 & 2.523 \\
		\hline
		7 & 3 & 3 & 2.523 \\
		\hline
		8 & 3.5 & 3.5 & 2.523 \\
		\hline
		9 & 4 & 4 & 2.534 \\
		\hline
		10 & 4.5 & 4.5 & 2.524 \\
		\hline
		11 & 5 & 5 & 2.534 \\
		\hline
	\end{tabular}
	\end{center}	
	\section{Graphs}
	\subsection{Forward Bias}
		\label{graph:fwd-bias}
		\includegraphics[width=1\linewidth]{"forward bias graph"}
	\subsection{Reverse Bias}
		\label{graph:rev-bias}
		\includegraphics[width=1\linewidth]{"reverse bias graph"}
		
	\section{Calculations}
	\begin{enumerate}
		\item Cut-in Voltage: To calculate the cut-in voltage, extend the characteristic line in graph \ref{graph:fwd-bias} backwards until it meets the x-axis. This point is the cut-in voltage. From graph \ref{graph:fwd-bias} we see that the cut-in voltage is \textbf{0.76V}.
		\item Static Resistance: To calculate the static resistance, we take a point on the graph, and use the following formula to calculate the resistance: \\ $ R_{DC}  =\dfrac{V_F}{I_F} = \dfrac{1.4V}{0.9A}$= \textbf{1.556 $ \Omega $}
		\item Dynamic Resistance: To calculate the dynamic resistance, take two points on the graph(fig. \ref{graph:fwd-bias}) and calculate the resistance with the following formula: \\ $ R_{AC} = \dfrac{\Delta V}{\Delta I}  = \dfrac{V_2 - V_1}{I_2 - I_1} = \dfrac{0.4}{1.65}  = $ \textbf{0.61$ \Omega $} \\
	\end{enumerate}
	
	\section{Precautions}
	\begin{enumerate}
		\item Make sure the circuit has a ground connection component.
		\item Check that the schematic is complete and the wires and components do not have gaps between them.
		\item Ensure that the correct components are selected.
	\end{enumerate}

	\section{Result}
	\begin{enumerate}
		\item Cut-in Voltage =\textbf{ 0.76 V}
		\item Static resistance =\textbf{ 1.556 V}
		\item Dynamic Resistance =\textbf{ 0.61 V}
	\end{enumerate}

	\section{Post-lab Questions}
	\begin{enumerate}
		\item \textbf{When diode acts like ideal switch?} Diode acts like an ideal switch in the forward bias condition.
		\item \textbf{What is the cut in voltage? Give typical values for Ge and Si.} Cut-in voltage, or knee voltage is the voltage of at point from where the current passing through the diode starts increasing rapidly. The typical value is 0.3V for Ge, and 0.7V for Si.
		\item \textbf{What is reverse saturation current?} Reverse saturation current is the current that flown even in reverse bias condition due to presence of minority charge carriers.
		\item \textbf{What is Dynamic and static resistance?} Dynamic resistance is the resistance offered by a diode when an AC voltage is applied across it. Static resistance is the resistance offered when direct current passes through it. 
		\item \textbf{Define potential barrier} The potential across the depletion region due to the electric field is called the potential barrier.
		\item \textbf{Define doping} Doping is the process of introducing electron-donor or electron-acceptor impurities in intrinsic semiconductors.
		\item \textbf{What is the effect of temperature on $ I_0 $} As temperature increases, $ I_0 $ increases as the number of minority charge carriers increase.
		\item \textbf{Define a Q point.} Q point is the voltage or current at a point in the circuit in its steady state.
		\item \textbf{Explain how the diode can acts as a capacitor} In reverse bias condition, a diode can act like a capacitor. In this case, as potential is applied, charges build up, which causes the diode to act as a capacitor.
	\end{enumerate}
\end{document}
 