%%%%%%%%%%%%%%%%%%%%%%%%%%%%%%%%%%%%%%%%%
% University/School Laboratory Report
% LaTeX Template
% Version 3.1 (25/3/14)
%
% This template has been downloaded from:
% http://www.LaTeXTemplates.com
%
% Original author:
% Linux and Unix Users Group at Virginia Tech Wiki 
% (https://vtluug.org/wiki/Example_LaTeX_chem_lab_report)
%
% License:
% CC BY-NC-SA 3.0 (http://creativecommons.org/licenses/by-nc-sa/3.0/)
%
%%%%%%%%%%%%%%%%%%%%%%%%%%%%%%%%%%%%%%%%%

%----------------------------------------------------------------------------------------
%	PACKAGES AND DOCUMENT CONFIGURATIONS
%----------------------------------------------------------------------------------------

\documentclass{article}

\usepackage[version=3]{mhchem} % Package for chemical equation typesetting
\usepackage{siunitx} % Provides the \SI{}{} and \si{} command for typesetting SI units
\usepackage{graphicx} % Required for the inclusion of images
\usepackage{natbib} % Required to change bibliography style to APA
\usepackage{amsmath} % Required for some math elements 

\usepackage{hyperref}
\hypersetup{
	colorlinks=true,
	linkcolor=blue,
	filecolor=magenta,      
	urlcolor=cyan,
}

\setlength\parindent{0pt} % Removes all indentation from paragraphs

\renewcommand{\labelenumi}{\alph{enumi}.} % Make numbering in the enumerate environment by letter rather than number (e.g. section 6)

%\usepackage{times} % Uncomment to use the Times New Roman font

%----------------------------------------------------------------------------------------
%	DOCUMENT INFORMATION
%----------------------------------------------------------------------------------------

\title{Determination of the Atomic \\ Weight of Magnesium \\ CHEM 101} % Title

\author{Pranav \textsc{Gade}} % Author name

\date{\today} % Date for the report

\begin{document}

\maketitle % Insert the title, author and date

\begin{center}
\begin{tabular}{l r}
Date Performed: & December 1, 2020 \\ % Date the experiment was performed
% Partners: & James Smith \\ % Partner names
% & Mary Smith \\
Instructor: & Dr. Somesh Kumar % Instructor/supervisor
\end{tabular}
\end{center}

% If you wish to include an abstract, uncomment the lines below
% \begin{abstract}
% Abstract text
% \end{abstract}

%----------------------------------------------------------------------------------------
%	SECTION 1
%----------------------------------------------------------------------------------------

\section{Objective}

To study the interface and functions of electronic circuit simulator software LTSpice and CircuitLab.

% If you have more than one objective, uncomment the below:
%\begin{description}
%\item[First Objective] \hfill \\
%Objective 1 text
%\item[Second Objective] \hfill \\
%Objective 2 text
%\end{description}

\subsection{Components Required}
\label{definitions}
\begin{description}
\item[\href{https://www.circuitlab.com/}{Circuit Lab}]
CircuitLab is a browser-based circuit simulation and design program.
\item[\href{https://www.analog.com/en/design-center/design-tools-and-calculators/ltspice-simulator.html}{LTSpice}]
LTSpice is a SPICE\footnote{Simulation Program with Integrated Circuit Emphasis} based simulation and analysis program.
\end{description} 
 
%----------------------------------------------------------------------------------------
%	SECTION 2
%----------------------------------------------------------------------------------------

\section{Background}

\subsection{LTSpice}
LTSpice is currently the most popular circuit simulation software. It provides a schematic design and entry feature used to design or load predesigned circuits. Its primary feature, however, is simulating given circuits to verify their integrity and predict their behavior. Finally, the waveform viewer is another important feature available. Moreover, LTSpice offers much better performance when compared to other similar simulators. This dramatically reduces the time required to run simulations. \\
LTSpice offers a very stable platform, as well as a robust and intuitive schematics editor. This, coupled with the vast library of passive devices available, results in an unparalleled user experience, for beginners and experts alike.
\subsection{Circuit Lab}
Circuit Lab is a convenient and accessible alternative to LTSpice. Since it runs inside your browser, it requires minimal set-up and can be used on almost any device with a browser. This, however, comes at the cost of not being as sophisticated as LTSpice. One  can design schematics very easily using their intuitive editor, and generate professional PDFs, diagrams, and plots within seconds. Simulating digital as well as analog circuits can be done in seconds. One of the best features, however, is the share option. One can make a schematic public with a click, and share it with others.

%----------------------------------------------------------------------------------------
%	SECTION 3
%----------------------------------------------------------------------------------------

\section{Procedure}

\begin{tabular}{ll}
Mass of magnesium metal & = \SI{8.59}{\gram} - \SI{7.28}{\gram}\\
& = \SI{1.31}{\gram}\\
Mass of magnesium oxide & = \SI{9.46}{\gram} - \SI{7.28}{\gram}\\
& = \SI{2.18}{\gram}\\
Mass of oxygen & = \SI{2.18}{\gram} - \SI{1.31}{\gram}\\
& = \SI{0.87}{\gram}
\end{tabular}

Because of this reaction, the required ratio is the atomic weight of magnesium: \SI{16.00}{\gram} of oxygen as experimental mass of Mg: experimental mass of oxygen or $\frac{x}{1.31}=\frac{16}{0.87}$ from which, $M_{\ce{Mg}} = 16.00 \times \frac{1.31}{0.87} = 24.1 = \SI{24}{\gram\per\mole}$ (to two significant figures).

%----------------------------------------------------------------------------------------
%	SECTION 4
%----------------------------------------------------------------------------------------

\section{Results and Conclusions}

The atomic weight of magnesium is concluded to be \SI{24}{\gram\per\mol}, as determined by the stoichiometry of its chemical combination with oxygen. This result is in agreement with the accepted value.

\begin{figure}[h]
\begin{center}
\includegraphics[width=0.65\textwidth]{placeholder} % Include the image placeholder.png
\caption{Figure caption.}
\end{center}
\end{figure}

%----------------------------------------------------------------------------------------
%	SECTION 5
%----------------------------------------------------------------------------------------

\section{Discussion of Experimental Uncertainty}

The accepted value (periodic table) is \SI{24.3}{\gram\per\mole} \cite{Smith:2012qr}. The percentage discrepancy between the accepted value and the result obtained here is 1.3\%. Because only a single measurement was made, it is not possible to calculate an estimated standard deviation.

The most obvious source of experimental uncertainty is the limited precision of the balance. Other potential sources of experimental uncertainty are: the reaction might not be complete; if not enough time was allowed for total oxidation, less than complete oxidation of the magnesium might have, in part, reacted with nitrogen in the air (incorrect reaction); the magnesium oxide might have absorbed water from the air, and thus weigh ``too much." Because the result obtained is close to the accepted value it is possible that some of these experimental uncertainties have fortuitously cancelled one another.

%----------------------------------------------------------------------------------------
%	SECTION 6
%----------------------------------------------------------------------------------------

\section{Answers to Definitions}

\begin{enumerate}
\begin{item}
The \emph{atomic weight of an element} is the relative weight of one of its atoms compared to C-12 with a weight of 12.0000000$\ldots$, hydrogen with a weight of 1.008, to oxygen with a weight of 16.00. Atomic weight is also the average weight of all the atoms of that element as they occur in nature.
\end{item}
\begin{item}
The \emph{units of atomic weight} are two-fold, with an identical numerical value. They are g/mole of atoms (or just g/mol) or amu/atom.
\end{item}
\begin{item}
\emph{Percentage discrepancy} between an accepted (literature) value and an experimental value is
\begin{equation*}
\frac{\mathrm{experimental\;result} - \mathrm{accepted\;result}}{\mathrm{accepted\;result}}
\end{equation*}
\end{item}
\end{enumerate}

%----------------------------------------------------------------------------------------
%	BIBLIOGRAPHY
%----------------------------------------------------------------------------------------

\bibliographystyle{apalike}

\bibliography{sample}

%----------------------------------------------------------------------------------------


\end{document}